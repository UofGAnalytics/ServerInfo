\documentclass{article}
\usepackage{handout}
\usepackage{etoolbox}
\usepackage{minted}
\usepackage[normalem]{ulem}


\usetikzlibrary{positioning}
\changeknitrcolours
\begin{document}
\course{Server Intro Session}
\title{Exercise}
\maketitle

This is a standforward exercise to allow you to get first hand experience using the servers, If you would rather replace the script here with a different script or a different task this is completely fine. We are simply trying to give you an opportunity to ask questions aboutt he use of the server.


\section*{Task}

Please run the Python/R (choose your language) script on one of the servers.
You can run this in any of the ways that we have demonstrated to you in this
session. If you get stuck feel free to ask questions of the PhD students and
staff in the room. 


The exercise has been designed so that you need to:

\begin{itemize}
  \item Transfer code onto the server
  \item (Potentially) Install/update libraries on the cluster
  \item Run a small amount of code 
  \item Transfer the code off the server.
\end{itemize}
 
The code to run can be found in this directory (with a little help with ChatGPT). 

Your workflow should be as follows:

\begin{enumerate}
  \item Connect to the VPN to gain access to the internal network
  \item Transfer the files to the server using scp from the commandline or a GUI application e.g.filevila or winscp.
  \item Select a Euclid to run the code on.
  \item Setup a basic R/Python environment to run the code 
  \item Run the code using one of the methods shown by our PhD students
  \item Transfer the results from the server back to your laptop. 
\end{enumerate}

Do not worry if you cannot finish this during the session, but if you think you
will need to use the server, please try in your time, and ask either one of the
PhD student who presented or me for help in running things. 

\subsection*{Python Users}

For those wishing to use Python rather than R, if you are doing this live in the session it fmight be easier to use the system Python (just to avoid setup times for anaconda. However in general we would recommend using an anaconda installation. 

\end{document}






